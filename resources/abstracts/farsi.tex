
% A B S T R A C T

\pagestyle{empty}

\begin{center}
\مهم{چکیده}
\end{center}
فواید بسیار استفاده از اینترنت، آن را به یکی از پرطرفدارترین روش‌های برقراری ارتباط و تبادل اطلاعات بدل نموده است. با افزایش روز افزون حجم دانش ذخیره شده در این شبکه‌ی جهانی، فرآیند دسترسی به بخش خاصی از مطالب که مدنظر کاربر است تبدیل به کاری دشوار شده است. بنابراین نیاز به ابزار جدید برای ساماندهی دسترسی و جستجو در این حجم انبوه از اطلاعات بیش از پیش حس می‌گردد. به همین منظور به مرور زمان، ابزارهای عامی برای جستجو در بین اطلاعات اینترنت، طراحی و پیاده‌سازی گردید. اما این ابزارهای کلی، نیازهای خاص کاربران را به طور کامل برطرف نمی‌کرد. یکی از نیازهای خاص کاربران، مقوله‌ی کاریابی است. در حال حاضر برای زبان فارسی موتور جستجوی کارآمدی برای پیدا کردن کسب‌وکار مناسب برای متقاضیان کار وجود ندارد.
\\
در این پروژه، سعی بر آن شد که با استفاده از سایت‌هایی که در زمینه‌ی کاریابی فعالیت دارند، یک سیستم یک‌پارچه‌ی جستجو پیاده‌سازی شود. برای این کار، نیاز به استخراج آگهی‌های کارفرمایان و کارنامک‌های متقاضیان از صفحات وب احساس می‌گردید. در این پروژه به بررسی سایت \لر{LinkedIn} که یکی از موفق‌ترین سایت‌های کاریابی انگلیسی زبان می‌باشد پرداخته شد و شمای داده‌ای کارنامک آن، استخراج گردید. علاوه بر آن، یک خزنده و یک موتور جستجوی متن باز، به گونه‌ای بهینه‌سازی گردید که امکان استخراج و شاخص‌بندی آگهی‌های کاریابی از سایت‌های فعال در این زمینه به صورت کارایی ممکن گردید. با این کار، دقت جستجو در بین آگهی‌ها بهبود یافت و اطلاعات غیرکاربردی صفحات وب، دور ریخته شد و حجم داده‌های ذخیره شده در شاخص‌ها، به میزان محسوسی کاهش پیدا کرد.

\مهم{کلمات کلیدی}:
آگهی کسب و کار، موتورهای جست‌وجو، خزنده‌ی \لر{Nutch}، شاخص‌بند \لر{Lucene}، داده‌کاوی.


\صفحه‌جدید