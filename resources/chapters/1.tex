
\فصل{پیش گفتار}
با افزایش روز افزون حجم دانش ذخیره شده به صورت دیجیتالی و در قالب‌های مختلف نظیر اخبار، صفحات وب، صفحات شخصی، مقالات علمی، کتاب‌ها، تصاویر، فایل‌های صوتی و تصویری و شبکه‌های اجتماعی، فرآیند جستجو به دنبال بخش خاصی از مطالب که مدنظر است و یافتن آن، تبدیل به کاری دشوار شده است. بنابراین نیاز به داشتن ابزار محاسباتی جدید که امکان سازماندهی، جستجو و فهم این حجم انبوه از اطلاعات را بدهد، بیش از پیش حس می‌شود.
\\
در حال حاضر، در مورد داده‌هایی که به صورت برخط ذخیره شده‌اند می‌توان از دو طریق جستجو و پیوند\زیرنوشت {link} صفحه، به مطلب مورد نظر دست یافت. به این صورت که می‌توان مطلب مورد نظر را در قالب واژگان کلیدی در یک موتور جستجو وارد کرد و در پاسخ به آن، مجموعه‌ای از اسناد مرتبط با عبارت جستجو را دریافت کرد. اما بعضاً ممکن است شخص جستجو کننده به جای جستجو به دنبال یک سند خاص، به دنبال مطالب در یک زمینه‌ی موضوعی خاص و ارتباط آن‌ها با یکدیگر باشد. در این صورت لازم است تا شخص جستجو کننده قبل از جستجو با استفاده از واژگان کلیدی، ابتدا زمینه را پیدا کرده و مطالب مرتبط با آن را مطالعه کند. این زمینه و اسناد مرتبط با آن ممکن است در گذر زمان نیز تغییر کنند. بنابراین استفاده از ساختار معنایی اسناد و طبقه‌بندی آن‌ها با استفاده از این ساختار، روشی دیگر برای کاوش و استفاده از اسناد است.
\\
در بسیاری از مجموعه‌های اسناد، به دلیل حجم بالای مطالب، نمی‌توان به طور کامل از قوای انسانی برای خواندن همه‌ی اسناد و پیدا کردن ساختار معنایی آن‌ها و جستجو به دنبال سایر اسناد مرتبط با استفاده از واژگان کلیدی استفاده کرد. به همین منظور روش مدل‌سازی موضوع\زیرنوشت {topic modeling} که مبتنی بر پردازش زبان طبیعی با استفاده از یادگیری ماشین است، به همراه جمع آوری و استخراج خودکار اطلاعات معرفی شده است.
\\
روش مدل‌سازی موضوع، یک مدل آماری برای یافتن عناوین استفاده شده در یک مجموعه با حجم بالا از اسناد، با استفاده از اطلاعات معنایی و ساختار معنایی نهان اسناد است. فرض اصلی روش‌های مدل‌سازی عناوین، تشکیل شدن هر سند از تعداد اندکی از عناوین است که در آن هر عنوان، دارای توزیع مشخص و مرتبط با موضوع از کلمات است. بنابراین کلماتی که در توزیع احتمال مربوط به هر عنوان به کار رفته در سند، با احتمال بالایی حضور داشته باشند، با احتمال بالایی جزء کلمات تشکیل‌دهنده‌ی سند نیز می‌باشند. بنابراین با استفاده از روش‌های آماری و به صورت مشابه، این الگوریتم‌ها، کلمات استفاده شده در متن را به منظور یافتن زمینه‌های معنایی اصلی به کار رفته در متن و همچنین یافتن ارتباط این زمینه‌ها و تغییرات آن در گذر زمان، بررسی می‌کنند.