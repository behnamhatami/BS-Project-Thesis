\فصل {پیاده سازی} 

\قسمت {مقدمه} 
در اين فصل روند طي شدن پروژه توضيح داده مي شود.
\\
در این پروژه، در ابتدا به بررسی‌های جامع در مورد وضع سایت‌های فعال کسب و کار در ایران، پرداختیم. البته بررسی‌ها به نمونه‌های داخلی محدود نگردید و بررسی‌هایی در مورد نمونه‌های خارجی موفق نیز انجام شد. با بررسی دقیق‌تر، سایت‌های هدف برای استخراج اطلاعات انتخاب شد. همچنین برای امکان ایجاد مشابه فارسی Linkedin و امکان استخراج اطلاعات طبقه بندی شده از Linkedin به بررسی شمای پایگاهی این سایت، اقدام نمودیم و تا حد خوبی، ساختار داده‌ای این سایت پرطرفدار خارجی را به دست آوردیم. سپس به مرحله‌ی پیاده سازی خزنده پرداختیم. در این مرحله، برای اجرای کار، نیاز به نوشته شدن، استخراج گر اختصاصی برای هر دامنه نیاز داشتیم. این استخراج گر، از هر صفحه‌ی آگهی داده‌های مفید آن را استخراج می‌نماید. این امر، باعث بهبود دقت جستجو در مراحل بعدی می‌گردد، زیرا داده‌های غیر مرتبط به آن آگهی را عملاً حذف می‌نماید. سپس با نوشتن افزونه برای تبدیل این داده‌های استخراج شده، به ساختاری سازگار با Lucene، امکان شاخص بندی داده‌ها را فراهم کردیم. سپس به بررسی میزان تأثیر گذاری این کار بر روی دقت نتایج جستجو پرداختیم.
\قسمت {پيش نيازها}
برای سازگار سازی Nutch با سایت‌های فعال در زمینه‌ی استخدام، نیاز به پیاده سازی افزونه‌ی خاص هر سایت است. علاوه بر این، نیاز است مکانیزمی تعبیه گردد که با توجه به سایت، افزونه‌ی خاص خود را پیدا کند و آن را اجرا نماید. برای چنین کاری، از الگوی کارخانه\زیرنوشت {Factory method} استفاده گردید. ساختار این الگو در زیر آمده است. این الگو با توجه به نوع سایت، که از روی آدرس سایت مشخص می‌شود، در بین افزونه‌هایی که خود را در سیستم ثبت نام کرده‌اند، جستجو می‌کند و در صورت پیدا کردن مورد مناسب آن را اجرا می‌کند. چنین ساختاری، اضافه کردن یک افزونه‌ی جدید برای یک سایت جدید را بسیار راحت می‌کند و از اضافه شدن تغییرات اجتناب می‌نماید. پس از آن باید داده‌های استخراج شده را برای شاخص بندی به Lucene اضافه نماییم، که توسط افزونه شاخص بندی انجام می‌پذیرد.

\قسمت {مراحل پیاده سازی} 
\زیرقسمت {بررسی و انتخاب سایت‌های هدف}
در این مرحله، با بررسی و جستجو در بین سایت‌های فارسی زبان که در زمینه‌ی استخدام و آگهی‌های مربوط به آن، فعالیت داشتند به دست آمد. به طور خاص، در مورد هر کدام، ویژگی‌های بارز استخراج شد. همچنین در این مرحله، به بررسی چند سایت خارجی که پیشرو در این زمینه بودند پرداخته شد. لیست سایت‌های بررسی شده در زیر آمده است:



\begin{table}
    \begin{tabular}{l|l|l|l|l}
    \hline
    ~ & ~ & ~ & ~ & ~ \\ \hline
    ~ & ~ & ~ & ~ & ~ \\
    ~ & ~ & ~ & ~ & ~ \\
    ~ & ~ & ~ & ~ & ~ \\
    ~ & ~ & ~ & ~ & ~ \\
    ~ & ~ & ~ & ~ & ~ \\
    ~ & ~ & ~ & ~ & ~ \\
    ~ & ~ & ~ & ~ & ~ \\
    ~ & ~ & ~ & ~ & ~ \\
    ~ & ~ & ~ & ~ & ~ \\
    \end{tabular}
\end{table}


\زیرقسمت {بررسی سایت LinkedIn}

\زیرقسمت {آماده سازی Nutch}

\زیرقسمت {پیاده سازی افزونه‌های تجزیه کننده}

\زیرقسمت {پیاده سازی افزونه‌ی شاخص بندی}

\قسمت {خلاصه}