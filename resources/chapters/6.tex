\فصل{نتایج}

\قسمت{مقدمه}
در این فصل قصد بررسی نتایج حاصل از انجام بخش پیاده سازی را در مقایسه با اجرای Nutch به صورت خام داریم.
\\
نرم افزار Nutch در هنگام بازیابی اطلاعات صفحات تمام اطلاعات را نگه می‌دارد و بدون حذف قسمت تبلیغات و کنار صفحه‌ها و تیتر بالای صفحه، اطلاعات را نگه می‌دارد. اما با استفاده از تجزیه گر موجود، هم در حجم اطلاعات ذخیره شده صرفه جویی گردید هم دقت جستجو در مراحل بعدی بهبود یافت. همچنین حذف منابع غیر مرتب از سایت با فیل‌تر کردن صفحات غیر مرتبط با آگهی نیز حذف گردید.
\\
در زمینه‌ی شاخص بندی، با کاهش حجم اطلاعات شاخص، هم موتور جستجو سریع‌تر نتایج خود را ارائه می‌دهد، هم اینکه کلمات نامرتبت را حذف می‌کند که دقت جستجو را افزایش می‌دهد.
\قسمت {محیط اجرای برنامه}
محیط اجرای برنامه، سیستم عامل اوبونتو\زیرنوشت {Ubuntu} نسخه $13.04$ بر روی لپ‌تاپ مک\زیرنوشت {Macbook} سال 2010 می‌باشد. مشخصات سخت افزاری شامل پردازنده 2 هسته‌ای $T9300$ با سرعت $2.2 Ghz$، با حافظه‌ی $4 GB$ و فضای دیسک ذخیره سازی $40 GB$ می‌باشد.
\\
در تمام مراحل از Nutch نسخه‌ی $2.2$ و Lucene نسخه‌ی $4.4$ استفاده گردید و به Nutch $3 GB $ حافظه اختصاص داده شد.

\قسمت {روش به کار رفته}
سه آزمایش برای جمع آوری داده توسط محیط بیان شده ارائه شد. اولین آزمایش با استفاده از Nutch برای بازیابی اطلاعات 3 سایت، تا عمق 5 و 50 صفحه‌ی اول در هر عمق انجام گردید. سایت‌های مورد آزمایش عبارتند از:
\شروع {فقرات}
\فقره {www.estekhtam.com}
\فقره {www.e-estekhdam.com}
\فقره {www.estekhdami.org}

\پایان {فقرات}

در آزمایش اول، برای سه سایت بالا، تعداد کلیدواژه‌های ذخیره شده را با حالت عادی مقایسه می‌کنیم.
در آزمایش دوم، برای سه سایت بالا، مدت زمان میانگین بازیابی را با 5 بار اجرای هر کدام، را اندازه گیری نمودیم.
در آزمایش سوم، به صورت دستی برای هر سه سایت، تعداد صفحات مرتبط در 20 صفحه‌ی پاسخ داده شده به عنوان صفحه‌ی مرتبط با هفت پرسمان استخدامی پرکاربرد که از سایت google استخراج شده بود، را به دست آوردیم. لیست پرسمان‌ها در جدول  ~\ref{queries} آورده شده است.

\شروع {لوح}
\وسط‌چین.
\شرح {پرسمان‌های به کار رفته برای بررسی دقت جستجو}
\برچسب {queries}
\شروع {جدول} {|c|}
\خط‌پر
پرسمان \\ \خط‌پر
استخدام بانک \\
استخدام شهرداری \\
استخدام نفت \\
استخدام بانک کشاورزی \\
استخدام وزارت دفاع \\
استخدام همشهری \\
استخدام نیروی انتظامی \\
\خط‌پر
\پایان {جدول}
\پایان {لوح}

\قسمت{نتایج و بحث}

\زیرقسمت{تعداد واژه‌ها}
در آزمایش اول، همان طور که می‌بینید هنگامی که جای استفاده از افزونه Nutch از افزونه‌ی طراحی شده استفاده گردید، به علت کاهش قسمت‌های تحت درگیر سایت، تعداد کلیدواژه‌ها به میزان 55 درصد تا 75 درصد کاهش پیدا کرده است. جزئیات هر سایت در جدول ~\ref{terms} آمده است.
\شروع {لوح}
\وسط‌چین.
\شرح {تعداد کلیدواژه‌های ذخیره شده بر اساس سایت}
\برچسب {terms}
\شروع {جدول} {c|c|c}
آدرس سایت & تعداد کلیدواژه در حالت عادی & تعداد کلیدواژه بهبود یافته \\ \خط‌پر
www.estekhtam.com & 5524 & 2202 \\
www.e-estekhdam.com & 2747 & 1266 \\
www.estekhdami.org & 2223 & 492 \\
\پایان {جدول}
\پایان {لوح}

\زیرقسمت{زمان بازیابی اطلاعات}
در آزمایش دوم، همان طور که می‌بینید هنگامی که از افزونه‌ی نوشته استفاده می‌گردد، زمان بازیابی اطلاعات به علت افزایش پردازش‌های انجام شده افزایش می‌یابد. به همین منظور نیاز به قدرت پردازشی و حافظه‌ی بیشتری برای کار نیاز دارد. جزئیات در جدول  ~\ref{times} آورده شده است.

\شروع {لوح}
\وسط‌چین.
\شرح {مدت زمان بازیابی بر اساس سایت}
\برچسب {times}
\شروع {جدول} {c|c|c}
آدرس سایت & زمان بازیابی در حالت عادی(s) & زمان بازیابی بهبود یافته(s) \\ \خط‌پر
www.estekhtam.com & $362.92$ & $375.2$ \\
www.e-estekhdam.com & $104.56$ & $105.12$ \\
www.estekhdami.org & $282.04$ & $316.37$ \\
\پایان {جدول}
\پایان {لوح}

\زیرقسمت{نتایج جستجو}
در آزمایش سوم، نتایج یک جستجوی خاص و دقت هر کدام قبل و پس از استفاده از افزونه با استفاده از Lucene مورد بررسی قرار گرفت. قابل توجه است که در روش به کار گرفته شده، سایت‌هایی که در خود از چند آگهی اطلاعات دارند دیگر در رده‌های بالا دیده نمی‌شود، به خصوص که هر کدام از صفحات، لیستی غیر مرتبط از آگهی‌ها را، در کنار خود دارند. جزئیات در جدول  ~\ref{precision} آورده شده است.


\شروع {لوح}
\وسط‌چین.
\شرح {میزان دقت جستجو به ازای هر سایت}
\برچسب {precision}
\شروع {جدول} {c|c|c}
آدرس سایت & دقت جستجو در حالت عادی & دقت جستجو بهبود یافته \\ \خط‌پر
www.estekhtam.com & 7 & 13 \\
www.e-estekhdam.com & 12 & 18 \\
www.estekhdami.org & 11 & 16 \\
\پایان {جدول}
\پایان {لوح}


\قسمت{خلاصه}
در این فصل، سعی بر آن داشتیم تا تأثیرات افزونه‌ی نوشته شده را بر قسمت‌های مختلف سیستم جستجو بررسی کنیم و ویژگی‌های مثبت و منفی آن را استخراج نماییم. در کل از نظر دقت جستجو و حجم موتور جستجو، استفاده از افزونه نوشته شده، مفید بوده است. اما استفاده از این افزونه نیاز به قدرت پردازشی بالاتری دارد که سخت افزار قوی‌تری را می‌طلبد.