\فصل {پیاده سازی} 

\قسمت {مقدمه} 
در اين فصل روند طي شدن پروژه توضيح داده مي شود.
\\
در این پروژه، در ابتدا به بررسی‌های جامع در مورد وضع سایت‌های فعال کسب و کار در ایران، پرداختیم. البته بررسی‌ها به نمونه‌های داخلی محدود نگردید و بررسی‌هایی در مورد نمونه‌های خارجی موفق نیز انجام شد. با بررسی دقیق‌تر، سایت‌های هدف برای استخراج اطلاعات انتخاب شد. همچنین برای امکان ایجاد مشابه فارسی Linkedin و امکان استخراج اطلاعات طبقه بندی شده از Linkedin به بررسی شمای پایگاهی این سایت، اقدام نمودیم و تا حد خوبی، ساختار داده‌ای این سایت پرطرفدار خارجی را به دست آوردیم. سپس به مرحله‌ی پیاده سازی خزنده پرداختیم. در این مرحله، برای اجرای کار، نیاز به نوشته شدن، استخراج گر اختصاصی برای هر دامنه نیاز داشتیم. این استخراج گر، از هر صفحه‌ی آگهی داده‌های مفید آن را استخراج می‌نماید. این امر، باعث بهبود دقت جستجو در مراحل بعدی می‌گردد، زیرا داده‌های غیر مرتبط به آن آگهی را عملاً حذف می‌نماید. سپس با نوشتن افزونه برای تبدیل این داده‌های استخراج شده، به ساختاری سازگار با Lucene، امکان شاخص بندی داده‌ها را فراهم کردیم. سپس به بررسی میزان تأثیر گذاری این کار بر روی دقت نتایج جستجو پرداختیم.

\قسمت {پيش نيازها}
برای سازگار سازی Nutch با سایت‌های فعال در زمینه‌ی استخدام، نیاز به پیاده سازی افزونه‌ی خاص هر سایت است. علاوه بر این، نیاز است مکانیزمی تعبیه گردد که با توجه به سایت، افزونه‌ی خاص خود را پیدا کند و آن را اجرا نماید. برای چنین کاری، از الگوی کارخانه\زیرنوشت {Factory method} استفاده گردید. ساختار این الگو در شکل ~\ref{fig:factory_method} آمده است. این الگو با توجه به نوع سایت، که از روی آدرس سایت مشخص می‌شود، در بین افزونه‌هایی که خود را در سیستم ثبت نام کرده‌اند، جستجو می‌کند و در صورت پیدا کردن مورد مناسب آن را اجرا می‌کند. چنین ساختاری، اضافه کردن یک افزونه‌ی جدید برای یک سایت جدید را بسیار راحت می‌کند و از اضافه شدن تغییرات اجتناب می‌نماید. پس از آن باید داده‌های استخراج شده را برای شاخص بندی به Lucene اضافه نماییم، که توسط افزونه شاخص بندی انجام می‌پذیرد.

\شکل‌پی‌ان‌جی{10}{مثالی از الگوی کارخانه}{factory_method}

\قسمت {مراحل پیاده سازی}
\زیرقسمت {بررسی و انتخاب سایت‌های هدف}
در این مرحله، با بررسی و جستجو در بین سایت‌های فارسی زبان که در زمینه‌ی استخدام و آگهی‌های مربوط به آن، فعالیت داشتند به دست آمد. به طور خاص، در مورد هر کدام، ویژگی‌های بارز استخراج شد. همچنین در این مرحله، به بررسی چند سایت خارجی که پیشروی در این زمینه، پرداخته شد. لیست سایت‌های بررسی شده در جدول  ~\ref{employSite-list}  آمده است.
\شروع {لوح}
\وسط‌چین
\شرح {لیست سایت‌های فعال کاریابی به زبان فارسی}
\برچسب {employSite-list}
\شروع {جدول}{c|p{3.5cm}|c|p{2.5cm}}
آدرس سایت & نوع تعامل با کاربر & به‌روزرسانی & نوع جستجو \\ \خط‌پر

estekhtam.com & ارسال توضیح\زیرنوشت {Comment} & هر روز & متن ساده\\
karyab.net & عضویت و ثبت کارنامک & هر روز & کارنامک، آگهی، نظرات  \\
banki.ir & عضویت و ثبت رزومه & هر 2 روز & کارجویان، شغل، تخصص و مکان\\
e-estekhdam.com & عضویت و ثبت رزومه & هر روز & مکان  \\
bazarekar.ir & عضویت و ارسال کارنامک & - & مقطع تحصیلی، رشته، جنسیت، استان و شغل \\
estekhdamnews.com & عضویت و ارسال رزومه & هر روز & متن ساده\\
unp.ir/jobs.php & خبرنامه با ایمیل & هر روز & متن ساده\\
estekhdamcenter.ir & عضویت کارجویان و  کارفرمایان، ثبت کارنامک و آگهی شغل & هر روز & متن ساده\\
estekhdami.org & دریافت آگهی و ارسال توضیح & هر روز & متن ساده \\


\پایان {جدول}
\پایان {لوح}

\زیرقسمت {بررسی سایت LinkedIn}
پس از بررسی سایت‌های فارسی، نوبت به بررسی سایت‌های موفق در زمینه‌ی کاریابی در زبان انگلیسی رسیدیم. یکی از بزرگ‌ترین شبکه های کاریابی، شبکه‌ی اجتماعی Linkedin می‌باشد. این شبکه در حال حاضر دارای ۲۰۰ میلیون کاربر فعال دارد. راه اندازی سایتی مشابه Linkedin به زبان فارسی و حتی استخراج کارنامک افراد از این سایت یکی از اهداف پروژه برای پیشنهاد کار می‌باشد. به همین منظور استخراج مدل داده ای این سایت به عنوان اولین قدم در زمینه‌ی به دست آوردن اطلاعات طبقه بندی شده از این سایت می‌باشد. در شکل ~\ref{fig:LinkedIn} شمای پایگاهی استخراج شده از این دامنه نشان شده است.

\شکل‌پی‌ان‌جی{9}{شماي داده اي سايت Linkedin}{LinkedIn}

\زیرقسمت {آماده سازی Nutch}
در این مرحله، باید ابتدا سایت Nutch را برای اجرا آماده ساری نمود. برای کار ما از نسخه‌ی ۲.۲ این نرم افزار استفاده نمودیم. این نسخه نسبت به نسخه‌های گذشته تغییرات بنیادی داشته است. به طور مثال می‌توان به ساختار ارتباط با پایگاه آن اشاره نمود. از این نسخه به بعد، با استفاده از تکنولوژی gora این امکان را ایجاد کرده است که Nutch مستقل از نوع پایگاه داده کار کند. به همین منظور در این پروژه از پایگاه داده‌ی mysql\زیرنوشت {www.mysql.com} استفاده گردید. برای راه اندازی اولین کارهایی که انجام شد تنظیم نوع پایگاه داده‌ی آن و ستون‌های جدول پایگاهی مربوطه بود.
\\
پس از این مرحله، باید خود نرم افزار را تنظیم می‌کردیم. در زیر تنظیمات کلی نرم افزار آمده است. در اینجا ما اسم عامل خزنده و لیست مجاز آن‌ها را مشخص کرده‌ایم. همچنین استفاده از sql نیز در اینجا مشخص شده است. تنظیمات آخر نیز تنظیمات رمزنگاری ارتباط شبکه‌ای را مشخص می‌کند.

\صفحه‌شکن.

\begin{latin} 
\begin{lstlisting}[style=listXML]
<?xml version="1.0"?>
<?xml-stylesheet type="text/xsl" href="configuration.xsl"?>

<configuration>
	<property>
		<name>http.agent.name</name>
		<value>EmploySpider</value>
	</property>
	<property>
		<name>http.robots.agents</name>
		<value>EmploySpider,*</value>
	</property>
	<property>
		<name>storage.data.store.class</name>
		<value>org.apache.gora.sql.store.SqlStore</value>
	</property>
	<property>
		<name>parser.character.encoding.default</name>
		<value>utf-8</value>
	</property>
</configuration>
\end{lstlisting}
\end{latin}
\زیرقسمت {پیاده سازی افزونه‌های تجزیه کننده}
در این قسمت، نیاز به پیاده سازی یک افزونه داشتیم که در آن با توجه به نوع صفحه، پردازشگر مخصوص به آن را اجرا نماید و قسمت های مفید صفحه را استخراج نماید. به طور کلی ساختار هر افزونه برای شناخته شدن در Nutch باید مطابق زیر باشد.
\begin{latin}
\begin{verbatim}
myPlugin/
  plugin.xml
  build.xml
  ivy.xml
  src/
    java/
      org/
        employ/
          nutch/
            parser/
              ...
\end{verbatim}
\end{latin}
در این ساختار، سه فایل plugin.xml نوع افزونه، نقاط گسترشی که پیاده سازی می‌نماید و کلاس‌های قابل اجرا برای هر کدام را مشخص می‌کند. فایل build.xml برای قابل اجرا کردن افزونه برای Nutch استفاده می‌گردد و فایل ivy.xml برای مشخص کردن پیش‌نیازهای این افزونه به کار می‌رود. 
\\
در شکل ~\ref {fig:parse.html} ساختار نمودار رده‌ی افزونه‌ی تولید شده نشان داده شده است. همان طور که می‌بینید برای پشتیبانی از چند دامنه، از الگوی کارخانه استفاده شده است.

\شکل‌پی‌ان‌جی {12} {ساختار افزونه‌ی تجزیه کننده} {parse.html}

\زیرقسمت {پیاده سازی افزونه‌ی شاخص بندی}
در این قسمت، ساختاری کاملاً مشابه همانند افزونه قبلی ساخته می‌گردد، با این تفاوت که نقاط گسترش متفاوتی را پیاده سازی می‌کند و در مرحله‌ی متفاوتی اجرا خواهد شد. ساختار و جزئیات این افزونه در شکل ~\ref {fig:indexer.usertags} آمده است.

\شکل‌پی‌ان‌جی {9} {ساختار افزونه‌ی شاخص بند} {indexer.usertags}

\قسمت {خلاصه}
در این فصل در ابتدا به بررسی روند کارهای انجام شده پرداخته شد. سپس هر کدام از روند‌ها به طور مفصل مورد بررسی قرار گرفت. از جمله روندهای بررسی شده می‌توان به بررسی جامع سایت‌های فارسی اشاره نمود. همچنین ساختار افزونه‌های Nutch نیز مورد بررسی قرار گرفت. در فصل آتی به بررسی ویژگی‌های کدهای پیاده سازی شده پرداخته خواهد شد.
