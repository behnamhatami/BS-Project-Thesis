\فصل{موتور جستجوی Nutch}

\قسمت{مقدمه}
پروژه‌ی \لر{Nutch} یک نرم‌افزار متن‌باز بر اساس زبان جاوا است که امکان خزش و جمع‌آوری طیف مختلفی از داده‌ها را، از یک شبکه‌ی داخلی، بخشی از اینترنت یا کل جهان گسترده‌ی وب را دارد
~\cite{Nutch-java}.
به طور کلی قبل از پیاده‌سازی \لر{Nutch} امکان تحلیل نتایج موتورهای جستجوی معروف، در برابر پرسمانی دلخواه، وجود نداشت و نتیجه‌ی جستجوی آن‌ها و نحوه‌ی رتبه‌بندی آن‌ها، با معیاری مناسب و منصفانه قابل مقایسه نبود. یکی از دلایل این امر، وجود الگوریتم‌های جستجوی اختصاصی و منابع بسته\زیرنوشت {Closed Source} در این شرکت‌ها بودند. البته دلیل چنین کاری، علاوه بر تمایل انحصار طلبی و رقابت، جلوگیری از سوء استفاده‌ی منتشرکنندگان هرزنامه‌ها، در بالا بردن رتبه‌ی یک دامنه‌ی خاص بود. پروژه‌ی \لر{Nutch} با متن‌باز بودن خود، سعی در برطرف کردن این معضل نمود. یکی از اهداف این موتور جستجو، ایجاد شفافیت و افزودن جزئیات به نحوه‌ی رتبه‌بندی صفحات وب بود. علاوه بر آن، ارائه‌ی یک موتور جستجوی جایگزین، برای افرادی که به وسیله‌ی محدود موتورهای جستجوی تجاری موجود امروزی، راضی نشده‌اند و دارای نیازهای خاص منظوره می‌باشند. همچنین ربات نظاره‌گر \لر{Nutch}، این امکان را به مدیران سایت‌ها می‌دهد که قسمت‌هایی از سایتشان که به وسیله‌ی این روش جمع‌آوری می‌گردد را تحت مدیریت و کنترل خود داشته باشند.
\\
معماری پروژه‌ی \لر{Nutch} به گونه‌ای طراحی شده است که، هم از لحاظ حجم جستجو و هم از لحاظ سرعت، قابل گسترش و بهبود می‌باشد. به همین منظور، استفاده از روش‌های توازی در بازیابی اطلاعات، به عنوان یکی از روش‌های بهینه‌سازی از لحاظ سرعت در پیاده‌سازی لحاظ شده است
~\cite{Nutch-java}.
بخش‌های اصلی \لر{Nutch} شامل سه قسمت اصلی واحد خزنده، واحد شاخص‌بندی و واسط جستجو بر روی داده‌ها است. خزنده‌ی \لر{Nutch} به گونه‌ای طراحی شده که بر روی هر شبکه‌ی داخلی یا خارجی کار می‌کند. اطلاعات بازیابی شده توسط این قسمت، در یک پایگاه‌داده به نام \لر{WebDb} برای استفاده‌های آتی ذخیره می‌شوند. خزنده، علاوه بر بازیابی و ذخیره سازی، با استفاده از نرم‌افزاری به نام \لر{Lucene} برای شاخص‌بندی اطلاعات بازیابی شده، استفاده می‌کند. از شاخص‌های به دست آمده، برای بازیابی اطلاعات به وسیله‌ی واسط جستجو استفاده خواهد شد.
\\ویژگی بارز \لر{Nutch} در برابر موتورهای جستجوی موجود، ساختار قابل گسترش آن می‌باشد. به طور مثال، \لر{Nutch} برای زمانی که نیاز به بازیابی یک یا چند دامنه‌ی خاص قابل استفاده است، تا زمانی که بعضی از اطلاعات یک دامنه را می‌خواهیم از صافی عبور دهیم، قابل استفاده است. \لر{Nutch} با استفاده از ساختار افزونه‌ای که توسط زبان نشانه‌گذاری\زیرنوشت {Markup language (XML)} به سیستم شناخته می‌شود امکان، چنین کاری را فراهم می‌آورد. چنین ساختاری که همانند ساختار \لر{Eclipse} می‌باشد، این امکان را ایجاد می‌کند تا بتوان بدون تغییر بنیادی در کد، به اصلاح رفتار برنامه اقدام نمود.

\قسمت {نحوه‌ی عملکرد}

\زیرقسمت {عملکرد کلی}

طبق شکل ~\ref{fig:Nutch_schema}، اجزای خزنده، شامل \لر{WebDb}، لیست واکشی، بازیاب‌ها و به‌روزرسان‌ها است. \لر{WebDb} یک پایگاه‌داده‌ی اختصاصی شده است که صفحات بازیابی شده به همراه پیوندهای ورودی و خروجی آن را در خود ذخیره می‌کند. همچنین در مورد هر صفحه، مجموعه‌ی کوچکی از اطلاعات، مانند آخرین زمان بازیابی، را ذخیره می‌کند. لیست واکشی، با استفاده از اطلاعات \لر{WebDb} تهیه می‌شود.  این لیست، شامل پیوند صفحاتی است که باید در این مرحله بازیابی گردد. بازیاب‌ها، با استفاده از  این لیست، صفحات را بازیابی نموده و \لر{WebDb} را متناسب با آن به‌روزرسانی می‌نمایند. در این مرحله، تغییر صفحه نسبت به بازیابی قبل، نیز برای کنترل تناوب بازیابی ذخیره می‌گردد. محتوای به دست آمده نیز برای جستجو استفاده می‌شود. این چرخه، به گونه‌ای طراحی شده است تا بتواند تا ابد اجرا گردد و همواره تصویری به روز از صفحات وب را ارائه دهد.
\\
زمانی که صفحات وب، بازیابی گردید، \لر{Nutch} امکان جستجو را با استفاده از \لر{Searcher} خود فراهم می‌کند. در واقع پس از این مرحله، ابتدا واحد شاخص‌بندی، محتوای استخراج شده را در لیست‌های شاخص وارون\زیرنوشت{Inverted Index} ذخیره می‌نماید. هر سند، به تعدادی ناحیه‌ی شاخص‌بندی تقسیم می‌گردد و هر کدام در یک رویه‌ی جداگانه شاخص‌بندی می‌شود. در نهایت، استخری\زیرنوشت{pool} از کارگزارها\زیرنوشت{Web Servers}، ارتباط بین کاربر و واحد جستجو را فراهم می‌کند. شکل کلی این رویه در شکل ~\ref{fig:Nutch_schema} آورده شده است.

\زیرقسمت {ساختار افزونه}
ساختار افزونه‌پذیری \لر{Nutch}، کامل شبیه ساختار \لر{Eclipse} می‌باشد
~\cite{Eclipse-plugin}.
 در واقع، \لر{Nutch} یک سیستم مرکزی برای کنترل یک مجموعه از ابزارها که با یکدیگر کار می‌کنند، می‌باشد، تا بتواند یک قابلیت را به آن اضافه نمود. بعد از مطالعه‌ی ساختار \لر{Eclipse} و اعمال آن به ساختار افزونه‌ی \لر{Nutch}، این نتیجه حاصل شد که مهم‌ترین المان‌های افزونه‌پذیری \لر{Nutch}، عبارتند از افزونه‌ها، نقاط گسترش پذیر و سیستم کنترل افزونه‌ها است. سیستم کنترل این امکان را به \لر{Nutch} می‌دهد که بدان، کارایی اضافه شود. این کارایی به وسیله‌ی یک افزونه، به سیستم اضافه می‌گردد. هر افزونه، در واقع یک المان گذاردنی\زیرنوشت {Pluggable} است که تعدادی نقاط گسترش را پیاده‌سازی می‌کند و این قابلیت‌ها، به وسیله‌ی سیستم کنترل مرکزی اجرا می‌شوند.

\شکل‌پی‌ان‌جی {10} {ساختار خزنده‌ی \لر{Nutch}} {Nutch_schema}

\قسمت {ساختار Lucene}

\زیرقسمت {عملکرد کلی}
کتاب‌خانه‌ی \لر{Lucene} به افراد، امکان افزودن شاخص‌بندی و جستجو در نرم‌افزارهای خود را می‌دهد
~\cite{Lucene-info}.
این کتاب‌خانه امکان شاخص‌بندی و جستجو در هر نوع داده‌ای را، تا زمانی که قابلیت تبدیل شدن به متن را داشته باشد، می‌دهد. این بدان معنی است که این کتاب‌خانه امکان جستجو در صفحات وب، فایل‌های \لر{PDF} و \لر{Words} را دارد. به این خاطر، \لر{Lucene} بهترین کتاب‌خانه‌ی برای نوشتن موتور جستجو است.

\زیرقسمت {شاخص‌بندی}
شاخص‌بندی، در واقع مرحله‌ی تبدیل متن به شاخص می‌باشد. شاخص‌ها در واقع خود، داده‌ساختاری‌اند که سرعت عملیات بازیابی اطلاعات را، بهبود می‌بخشند. شاخص‌ها، جزء المان‌های اصلی  \لر{Lucene} می‌باشند.
\\
برای شاخص‌بندی داده در این سیستم، داده باید به جویباری\زیرنوشت {Stream} از تکه‌های متنی تبدیل گردند. بعد از آن، \لر{Lucene}، داده‌ها را با تکه تکه کردن جویبار داده و اجرای عملیاتی بر روی آن، آن را برای شاخص‌بندی آماده می‌کند. به طور مثال، یکی از این عملیات می‌تواند، کوچک‌سازی حروف، برای از بین بردن حساسیت جستجو به بزرگی و کوچکی باشد. این مرحله به مرحله‌ی تحلیل معروف است. بعد از اینکه جویبارها، تحلیل گردیدند، داده‌ها آماده برای اضافه شدن به شاخص‌ها می‌باشند. مرحله‌ی شاخص‌بندی در شکل ~\ref {fig:lucene} متصور شده است.

\شکل‌پی‌ان‌جی {10} {ساختار \لر{Lucene}
~\cite{Text-search}} {lucene}

این کتاب‌خانه، روشی نوآورانه برای نگهداری شاخص‌ها به کار می‌برد. این کتاب‌خانه مستقل از تعداد شاخص‌ها، برای هر شاخص از چند قطعه استفاده می‌کند. استفاده کردن از چند قطعه، اضافه کردن یک سند تازه شاخص‌بندی شده را به وسیله‌ی اضافه کردن آن به کوچک‌ترین قطعه که تازه ساخته شده است و ترکیب آن قطعه با بقیه‌ی قطعات به صورت متناوب، تسریع می‌بخشد. این رویه، اضافه کردن سند را بسیار کارا می‌کند، زیر شاخص‌هایی که در هر لحظه تغییر می‌کند را به شدت کاهش می‌دهد.
\\
بعضی از سیستم‌های بازیابی، برای اضافه شدن یک سند، نیاز به بروز رسانی تمام شاخص دارند، اما \لر{Lucene}، به علت پشتیبانی از شاخص‌بندی افزایشی، چنین مشکلی را ندارد. چنین امری بدان معناست که \لر{Lucene} امکان جستجو بر روی سند را بلافاصله بعد از شاخص‌بندی شدن آن بدون نیاز به بروز رسانی تمام شاخص، فراهم می‌سازد.

\زیرقسمت {تحلیل‌گر}

همان طور که در قسمت قبل، بحث شد، تحلیل یکی از مهم‌ترین مراحل، شاخص‌بندی است. این مرحله، داده‌های متنی را به یکی از اساسی‌ترین نمایش شاخص یعنی واژه‌های بنیادی\زیرنوشت {Terms}، تبدیل می‌نماید. این واژه‌های بنیادی برای تطابق پرسمان‌ها در مرحله‌ی جستجو با اسناد استفاده می‌شود.

\قسمت {خلاصه}
در این فصل، ساختار \لر{Nutch} به همراه مکانیزم افزونه‌پذیری مورد مطالعه قرار گرفت. همچنین ساختار شاخص‌بندی \لر{Lucene} و تحلیل‌گر آن نیز به اختصار توضیح داده شد. در فصل آتی، پیاده‌سازی افزونه‌ها و تغییرات اعمال شده توضیح داده خواهد شد.
