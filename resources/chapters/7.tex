\فصل {نتیجه‌گیری} 

\قسمت {خلاصه} 
استخراج مطالب اصلی صفحات وب، یکی از چالش‌برانگیزترین کار برای به دست آوردن اطلاعات مفید از صفحات وب می‌باشد. یکی از روش‌های اجرای این کار، نوشتن تجزیه‌گر اختصاصی برای یک سایت می‌باشد. اما به علت تنوع سایت‌ها، امکان اجرای دقیق کار، برای طیف وسیعی از سایت‌ها وجود ندارد، اما امکان اجرای این کار برای مجموعه‌ای محدود از سایت‌ها قابل اجرا است. به همین منظور استفاده از تجزیه‌گر اطلاعات برای مجموعه‌ای از سایت‌های کاریابی پیشنهاد گردید. گرچه این کار، دقت نوشتن تجزیه‌گر اختصاصی برای هر سایت را ندارد، اما از استخراج‌کننده‌های مطالب اصلی و کلی، معمولاً بسیار کاراتر می‌باشد. از طرفی روش به کار رفته صرفه جویی در حافظه‌ی مصرفی و افزایش سرعت جستجو را تضمین می‌کند. 
\\
آزمایش‌های انجام شده، نشان داد که استفاده از این روش معایب و فواید خاص خود را دارد. از جمله فواید آن بهبود دقت جستجو می‌باشد. زیرا استخراج مطالب مفید هر صفحه و حذف مطالب نامرتبط و تبلیغ گونه در هر صفحه در دقت جستجو تأثیر بسزایی دارد. از طرفی، اجرای افزونه باِعث کاهش سرعت اجرای بازیابی می‌گردد و تجزیه‌ی هر صفحه، حافظه‌ی مصرفی آن لحظه را به علت داده‌ساختار به‌کارگرفته شده بالا می‌برد. در مقابل آن، کاهش حجم حافظه‌ی ذخیره‌شده‌ی مصرفی، نیز به شدت محسوس بود. این ویژگی در شاخص‌بندی یک نکته‌ی مثبت است. 

\قسمت {کارهای آینده} 
در گام بعدی، نیاز داریم افزونه‌ی نوشته شده را برای سایت‌های مختلف موجود سازگار کنیم. همچنین سطح اطلاعات سیستم را با سازگار کردن به استخراج کارنامک افراد از سایت‌های ارائه دهنده‌ی خدمات در این زمینه بهبود بخشیم. همچنین امکان بهبود سرعت استخراج اطلاعات، را با تغییر مکانیزم استخراج و استفاده از الگوهای زبان منظم انجام دهیم. همچنین افزونه می‌تواند برای سایت‌هایی که سازگار نیست یک الگوریتم عادی برای استخراج محتوای اصلی استفاده نماید.